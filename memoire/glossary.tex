% editer les définitions du glossaire

% \makeglossaries

% \newglossaryentry{latex}{
%     name=LaTeX,
%     description={Un système de composition de documents très utilisé pour la production de documents techniques et scientifiques}
% }

% \newglossaryentry{pdflatex}{
%     name=pdfLaTeX,
%     description={Une version de LaTeX qui produit directement des fichiers PDF}
% }

% \newglossaryentry{raycasting}
% {
%     name={ray casting},
%     description={Une forme plus simple de ray tracing utilisée dans des jeux tels que Wolfenstein 3D et Doom, qui tire un seul rayon et s'arrête lorsqu'il atteint une cible.}
% }

% \newglossaryentry{raymarching}
% {
%     name={ray marching},
%     description={Méthode de lancer de rayon qui utilise des champs de distance signés (SDFs) et généralement un algorithme qui représente une sphère qui "marche" sur les rayons de façon incrémentale jusqu'à ce qu'elle atteigne l'objet le plus proche.}
% }

% \newglossaryentry{raytracing}
% {
%     name={ray tracing},
%     description={Une version plus sophistiquée de la projection de rayons qui émet des rayons, calcule les intersections rayon-surface et crée récursivement de nouveaux rayons à chaque réflexion.}
% }

% \newglossaryentry{pathtracing}
% {
%     name={path tracing},
%     description={Un type d'algorithme de traçage de rayon qui tire des centaines ou des milliers de rayons par pixel au lieu d'un seul.Les rayons sont tirés dans des directions aléatoires à l'aide de la méthode Monte Carlo, et la couleur finale du pixel est déterminée à partir de l'échantillonnage des rayons qui atteignent la source lumineuse.}
% }

