\chapter{FoxDot - algorave}\label{appendix:foxdot}
\todo{migrer foxdot doc plutôt vers annexes}

\section{Introduction}
\subsection{Lien avec supercollider}
\subsection{Les synthdefs, définition}

\section{Présentation de l'éditeur}
\todo{ajouter les Player.get\_attributes() etc pour avoir les infos}
\begin{itemize}
    \item évaluation du code
    \item interactivité
    \item consulter la doc
\end{itemize}

\section{Player objects}
\subsection{Player attributes}
Player.get\_attributes()
\subsection{Player effects}
\subsection{Jouer des samples}
\section{Patterns}
\subsection{Méthodes}
\subsection{Fonctions}
\subsection{Générateurs}
\subsection{PGroups}

\section{timevars}
\subsection{vue d ensemble}
\subsection{différents types, var, PVar}

\section{Manipulation algorithmique avec every}

\section{écrire ses propres fonctions}
\url{https://crashserver.fr/tutorial-write-your-own-foxdot-custom-function-with-python/}

\section{player keys}

\section{groupes}

\section{roots, Pvar}

\section{méthode play()}

\section{Clock}

\section{Plus loin}
\subsection{envoyer de l osc}
\subsection{envoyer du MIDI}
\todo{décrire la structure du MIDI}
\subsection{écrire ses propres synthdefs}

\section{Bonus: renardo}

