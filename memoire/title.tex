\begin{titlepage}
\begin{center}

% \textsc
{\LARGE \textbf{Université Paris 8}}\\[1.5cm]

% \textsc
{\Large \textbf{Master Création Numérique}}\\[0.25cm]

% \textsc
{\Large parcours: \textit{Arts et Technologies de l'Image Virtuelle}}%\\[1.25cm]

\vfill
% Title

% Comment musique et visuels peuvent interagir lors d'une performance live demoscene? invisible vers visible
% / musique algo / réseau / interaction / vers une interaction sans filtres entre live coders/ f5 pour ceux qui ont pas le live\\

% \todo{en intro dire pour quoi j'ai choisi ce titre}
\HRule \\[0.2cm]
{\LARGE \bfseries 
Vers une interaction authentique entre \textit{livecoders}}
\\[0.4cm]
{\Large \textit{Forward The Revolution}}
\HRule \\[1.5cm]
% Author and supervisor
%\begin{minipage}{0.4\textwidth}
%\begin{flushleft} \large
% \emph{Auteur:}\\
\HRule \\[0.4cm]
{\large \textbf{Guillaume Cournet}}\\
\HRule \\[1.5cm]

%\textsc{Auteur}\\
%\textsc{Auteur}\\
%\textsc{Auteur}
%\end{flushleft}
%\end{minipage}
%\begin{minipage}{0.4\textwidth}
%\begin{flushright} \large
%\emph{Client:} \\
%Prénom \textsc{Nom}\\
%\emph{Référent:} \\
%Prénom \textsc{Nom}
%\end{flushright}
%\end{minipage}

\vfill

% Bottom of the page
% Upper part of the page. The '~' is needed because only works if a paragraph has started.
\begin{comment}
\begin{figure}[!h]
\begin{center}
%taille de l'image en largeur
%remplacer "width" par "height" pour régler la hauteur
\includegraphics[width=15cm]{./images/foxdot_touch.jpg}
\end{center}
%légende de l'image
\caption{logo ati}
\end{figure}
\end{comment}

\includegraphics[width=0.60\textwidth]{images/logos/ATI_LogoNoir02.png}~\\[1cm]

{\large \textbf{Mémoire de Master 2, 2023 - 2024}}

% {\large \today}
\end{center}
\end{titlepage}