\begin{titlepage}
    \begin{minipage}[t][\textheight]{\textwidth}
        \vspace*{\fill}
        \begin{flushleft}         
            \Huge\textbf{Remerciements}
            \addcontentsline{toc}{chapter}{Remerciements}
        \end{flushleft}
        \vspace{1cm}
        \begin{flushleft}

            % \todo{jeff conseils/avisés toujours justes pour le mem que j'aurais du écouter + manu notre mascotte pour sagesse africaine de la cravate + massi}
            Ces remerciements font écho, d'une certaine manière, aux \textit{greetings}\footnote{Dans la culture de la \textit{demoscene}, les \textit{greetings} sont des salutations, souvent inclues dans les \textit{demos}, adressées à d'autres groupes de la \textit{demoscene}, à des personnes spécifiques ou à la communauté dans son ensemble.} de la \textit{demoscene}.
            
            Je tiens tout d'abord à exprimer mes sincères remerciements à l'équipe pédagogique d'ATI pour son soutien et sa disponibilité tout au long de mon parcours à ATI. Un merci particulier à Farès Belhadj pour m'avoir donné l'opportunité d'assister à son cours de programmation graphique en tant qu'auditeur libre. Grâce à ses explications claires et détaillées, j'ai pu approfondir ma compréhension des concepts mathématiques sous-jacents à la création d'une scène 3D. Un merci particulier aussi à Alain Lioret pour m'avoir subtilement montré la voie vers l'exploration de l'art numérique. Un grand merci à Jeff Jego pour son soutien constant tout au long de cette année de M2, ainsi que pour ses conseils avisés et pertinents auxquels a posteriori j'aurais dû accorder plus d'attention.
            
            % \todo{en vérité j'aimerais adresser mes remerciements à chacun et à chacune (notez le non emploi du langage inclusif)}
            
            Un énorme merci également à l'ensemble de ma promotion pour leur bonne humeur et leur accueil chaleureux. Je tiens particulièrement à adresser mes remerciements à Loïck pour son aide précieuse dans la gestion du temps de rédaction de ce mémoire, ainsi qu'à Garvey pour son « dématrixage » artistique.
            
            % \todo{the last but not the least}
            
            Je ne saurais aussi passer sous silence l'apport essentiel d'Antoine Boellinger, sans qui je n'aurais jamais découvert l'existence même des \textit{shaders}. Mes remerciements vont également à tous les membres du Cookie Collective pour leur accueil bienveillant et leur partage de connaissances. Je souhaite notamment exprimer ma reconnaissance envers les animateurs d'atelier au Fuz : z0rg pour ses ateliers de \textit{creative coding}, Élie Gavoty pour ses sessions sur FoxDot, et Jules pour ses enseignements sur SuperCollider. Je n'oublie pas non plus Pérégrine pour son exigence mathématique et son dévouement dans la rédaction de la précieuse documentation du wiki du Fuz.
            
            Enfin, un grand merci à ma famille ainsi qu'à mon ami Nissim pour leur soutien, leurs encouragements et leur implication dans la relecture de ce mémoire, qui s'est avérée être indispensable.
        \end{flushleft}
        \vspace*{\fill}
    \end{minipage}
\end{titlepage}

% \cleardoublepage