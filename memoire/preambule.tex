\begin{titlepage}
    \begin{minipage}[t][\textheight]{\textwidth}
        \vspace*{\fill}
        \begin{flushleft} 
            \Huge\textbf{Préambule}
            \addcontentsline{toc}{chapter}{Préambule}
        \end{flushleft}
        \vspace{1cm}
        \begin{flushleft}

Le recours délibéré aux anglicismes mérite d'être souligné. Ces termes anglais ont été sélectionnés pour leur précision et leur pertinence dans le domaine du \textit{live coding}. L'usage de ces expressions étrangères s'inscrit dans le souhait de demeurer fidèle au langage communément utilisé dans le milieu de la \textit{demoscene}, où l'influence de l'anglais est prépondérante. J'ai veillé à ce que ces mots soient mis en \textit{italique} dans le texte.

Par ailleurs, il me semblait important de préciser ma méthodologie quant à l'utilisation de l'intelligence artificielle avec ChatGPT 3.5. Cette dernière a été principalement sollicitée pour reformuler certains paragraphes, tant du point de vue orthographique que du rythme des phrases, et aussi dans le but d'éviter les répétitions. Cette interaction avec l'IA peut être assimilée à un dialogue, similaire à une partie de ping-pong, jusqu'à ce que le résultat satisfaisant soit obtenu. En outre, j'ai également utilisé l'IA pour me suggérer des titres de paragraphes lorsque j'étais en panne d'inspiration. Cependant, j'ai également pris conscience des risques liés à la dépendance à l'IA. À titre d'anecdote, dans le but d'économiser du temps, j'ai essayé de résumer une vidéo d'une conférence portant sur l'histoire de la \textit{demoscene}, en fournissant à l'IA les sous-titres corrects de la vidéo. Cependant, après avoir revu la vidéo ultérieurement, j'ai constaté que l'IA ignorait des thèmes importants et mélangeait les dates ainsi que les titres de \textit{demos}. La leçon que j'en ai tirée est qu'il est essentiel de toujours vérifier les informations produites par l'IA, comme le souligne d'ailleurs l'interface de ChatGPT : \textit{ChatGPT can make mistakes. Consider checking important information}.
        \end{flushleft}
        \vspace*{\fill}
    \end{minipage}
\end{titlepage}

\cleardoublepage
